\chapter{Présentation detaillée du stage}

\begin{itemize}

 \item \textbf{Titre du stage} :

 Etude,développement, mise en place  et suivi d'une solution d'acquision et de stockage de données sur le site expériemental IRSTEA de Montoldre dans le cadre du Projet OPEROSE
 \item \textbf{Maître de stage} : François Pinet
 \item \textbf{Lieu du stage} : site des Cézaux.

\item \textbf{Sujet de stage détaillé}: 

Ce stage s’inscrit dans le cadre du projet OPEROSE, Organisation Opérationnelle du Challenge ANR ROSE (Robotique et capteurs au service d’Ecophyto). 

Le Challenge ANR ROSE a pour objectif d’encourager le développement de solutions innovantes autonomes en matières de désherbages intra-rang de grandes cultures à fort écartement (maïs, tournesol, etc…) et des cultures légumières de plein champ afin de réduire l’usage des herbicides.

L’objectif est donc, de contribuer au développement d’une solution de collecte, de stockage de données acquises sur les parcelles agricoles mises à disposition des participants pour la compétition.

Il s’agit, de constituer une base de données (modélisation et développement) pour stocker et gérer les données collectées au moyen de capteurs. Cette base de données sera la source d’un dispositif d’aide à la décision servant à définir quand intervenir sur les parcelles mis à la disposition de chacun des participants du challenge. Cette base de données est développée en fonction des besoins et des capteurs identifiés et devra être rendu disponible pour une consultation à distance.

Une partie du stage est aussi consacrée au projet CASPAR MULTIPASS. Sur cette partie, Il s’agit de faire une étude sur les méthodes d’anonymisation des données. L’objectif de cette étude est de garantir la confidentialité sur des données à caractères personnelles pouvant ressortir des données collectées.

\end{itemize}
