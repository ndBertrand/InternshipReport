\chapter{Structure prévisionnel du rapport de stage}
\paragraph{Chapitre 1 : Présentation de l'entreprise}
Dans ce chapitre une brève présentation(qui se retrouve dans ce document) de l'IRSTEA ainsi qu'une présention de l'environement de travaille y seront rapportés.
\paragraph{}
Dans la partie présentation de l'IRSTEA, figurera  : 
\begin{itemize}
 \item les domaines de recherche de l'IRSTEA
 \item La vision est l'ambition du laboratoire
 \item Quelques chiffres et résultat
 \end{itemize}
\paragraph{}
dans la partie présentation de l'environement de travaille figurera : 
\begin{itemize}
 \item Présentation du centre IRSTEA de Clermont-Ferrand et Montoldre
 \item Projet innovant en cours et à venir
 \end{itemize}



\paragraph{Chapitre 2 : Contexte et travail effectué}
Ce chapitre présentera tous les projets sur lesquelles j'ai eu la chance de travailler tout le long de mon stage
C'est en dans cette partie que je résumerai les phases préparatoir pour chaque projets, l'analyse des sujets, les approches et les méthodologies adoptés, etc... . Des sous partie comme la présentation de la problématique, la présentation de l'existant, ainsi que la solution apporté figureront également dans ce chapitre.

\paragraph{Chapitre 3 : Bilan}
Ce chapitre se fera vera la fin et constituera un auto-évaluation au travail accomplie. Ce chapitre se déclinera sous 2 partie principale. Un bilan technique et un bilan personnel. Le bilan technique exposera l'ensemble des réussite et echecs au niveau technique, l'apport technique que j'aurais acquis durant mon stage ainsi qu'une proposition d'évolution du travail accomplie. Le bilant personnel dénotera mon expérience ausein d'un environement professionnel. Cette partie couvrira l'expérience que j'aurais acquise et une projection sur mon orientation professionel futur sera fait.






