\chapter{Déscription de l'entreprise}
\section{Présentation de l'entreprise}
\paragraph{}
Irstea (Institus national de recherche en sciences et technologies pour l'environnement et l'agriculture) est un établissement public à caractère scientifique  et technologique réparties en France dans 9 centres à savoir:  
\begin{itemize}
 \item le centre d'Aix-en-Provence
 \item le centre d'Antony : siège social
 \item le centre de Bordeaux
 \item le centre de Clermont-Ferrand
 \item le centre de Grenoble
 \item le centre de Lyon-Villeurbanne
 \item le centre de Montpellier
 \item le centre de Nogent
\end{itemize}

Ses activités de recherches et d'expertises sont tournées à la fois vers un questionnemet scientifique et sociaétal dans le le domaine de l'environnement et de l'agriculture. 

\paragraph{}
Un réseau de 19 unités de recherche(UR) et unités mixtes de recherche(UMR) forme un lieu d'interaction permanante avec des acteurs de l'enseignement supérieur, du secteur économique et autres organismes ainsi que des partenaires étrangers.

\section{Mon environnement de Travail}
\paragraph{}
Au sein de l'IRSTEA, je fait parti d'une équipe COPAIN, ,elle même rattachée à une unité mixte de recherche en technologies et systèmes d'informations pour les agrosystèmes (UMR TSCF) comptant environ 60 agents  réparties sur le pôle scientifique et universitaire des Cézaux à Aubière(63) et le site de recherche et d'expérimentation de Montoldre(03).
\paragraph{}
Je travaille ainsi en équipe avec un certain degré d'autonomie. Régulièrement, des réunions d'équipes ou de projets sont organisés par vidéo conférence ou par téléphone avec les personnes impliqués qu'elles soient sur place au géographiquement éloignées. Pour le projet OPEROSE, je suis souvent amené à échanger avec le personnel de Montoldre dont un stagiaire travaillant sur le même sujet que moi.
\paragraph{}
Généralement, depuis le début de mon stage, au moins une fois par semaine, je m'entretiens avec mon maître de stage pour faire le point. Je lui présente mon état d'avancement ainsi que les obstacles rencontrés, et lui me conseille sur la manière d'aborder le problème sous un angle différent.
\paragraph{}
Un bureau avec un poste de travaille(ordinateur), et un poste téléphonique et tout ce qu'il faut pour travaillé m'ont été attribué. Un service d'assistance Informatique est à la disposition d'un tout le personnel en cas de besoin. Au tout début du stage, nous avons également eu une rapide présentation des différents outils utilisés au sein de l'IRSTEA. Une formation GIT, pour un travaille collaboratif, a même été dispensé pour toute personne qui le souhaitait 
\paragraph{}
Au-delà de deux mois de stage et dans la limite des six mois, tous les stagiaires bénéficient d'une journée de congé par mois de présence, soit 5 jour dans mon cas. J'ai aussi l'autorisation de m'absenter du fait d'obligations attestés par mon établissement d'enseignement avec un accord préalable de mon maître de stage, du directeur d'unité et du directeur régional.
\paragraph{}
Irstea subventionne aussi la restauration du personnel et des stagiaire afin de faciliter la prise de repas du midi à proximité du lieu de notre travaille. Par le biais de l'Association Cézamis, le personnelle d'Irstea ainsi que les stagiaire bénéficient également des réductions sur les places de Cinéma, des activité comme la Gym, le Qi cong, Football etc....