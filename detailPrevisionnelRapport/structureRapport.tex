\chapter{Structure prévisionnelle du rapport de stage}
\paragraph{Titre prévisionnel : }
Modélisation et développement d'une base de données météorologiques et étude sur l'anonymisation des données

\paragraph{Structure prévisionnelle}
\begin{itemize}
    

\item {\textbf{Chapitre 1  Cadre et modalité du stage : }}
Dans ce chapitre je ferai une brève présentation(qui se retrouve également dans ce document) de l'IRSTEA ainsi qu'une présentation de l'environnement de travaille. je présenterai également le contexte du stage.
\\
Dans la partie présentation de l'IRSTEA, figurera  : 
\begin{itemize}
 \item les domaines de recherche de l'IRSTEA
 \item La vision est l'ambition du laboratoire
 \item Présentation du centre IRSTEA de Clermont-Ferrand et Montoldre
 \end{itemize}
et dans la partie contexte du stage, figurera : 
\begin{itemize}
    \item la présentation de l'existant
    \item les résultats attendu
\end{itemize}

\item {\textbf{Chapitre 2 : Contexte et travail effectué}}
Ce chapitre présentera tous les projets sur lesquelles j'ai eu la chance de travailler tout le long de mon stage.
C'est en dans cette partie que je résumerai les phases préparatoire pour chaque projets, l'analyse des sujets, les approches et les méthodologies adoptés, etc... . Des sous partie comme la présentation de la problématique, la présentation des solution apportées, les outils utilisés, la documentation figureront également dans ce chapitre.

\item {\textbf{Chapitre 3 : Bilan}}
Ce chapitre se fera vers la fin et constituera un auto-évaluation au travail accomplie. Il se déclinera sous 2 parties principales. Un bilan technique et un bilan personnel. Le bilan technique exposera l'ensemble des réussite et échecs au niveau technique, l'apport technique que j'aurais acquis durant mon stage ainsi qu'une proposition d'évolution du travail accomplie. Le bilant personnel dénotera mon expérience au sein d'un environnement professionnel. Cette partie couvrira donc  l'expérience acquise et une projection sur mon orientation professionnel.
\end{itemize}





